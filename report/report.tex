% This is samplepaper.tex, a sample chapter demonstrating the
% LLNCS macro package for Springer Computer Science proceedings;
% Version 2.20 of 2017/10/04
%
\documentclass[runningheads]{llncs}
%
\usepackage{graphicx}
\usepackage[hidelinks,breaklinks=true,backref=page]{hyperref}

\usepackage[spanish]{babel}
\usepackage[utf8]{inputenc}
% Used for displaying a sample figure. If possible, figure files should
% be included in EPS format.
%
% If you use the hyperref package, please uncomment the following line
% to display URLs in blue roman font according to Springer's eBook style:
% \renewcommand\UrlFont{\color{blue}\rmfamily}

\begin{document}
%
\title{Argumentación en Prensa Cubana}
%
%\titlerunning{Abbreviated paper title}
% If the paper title is too long for the running head, you can set
% an abbreviated paper title here
%
\author{Luis Ernesto Ibarra Vázquez\inst{1} \and
Luis Enrique Dalmau Coopat\inst{2} \and 
Adrián Hernández Pérez\inst{3}}
%
\authorrunning{L. Ibarra, L. Dalmau, A. Hernández}
% First names are abbreviated in the running head.
% If there are more than two authors, 'et al.' is used.
%
\institute{Universidad de La Habana}
%
\maketitle              % typeset the header of the contribution
%
\begin{abstract}

El proyecto se basa en el estudio de la argumentación en los periódicos cubanos. 
El objetivo principal es el análisis de las estructuras argumentativas que aparecen 
en ellos para conocer cuáles son las más usadas, cuáles son los argumentos expresados, 
entre otras estadísticas. En una primera etapa se concentrará en la segmentación y en 
la clasificación entre argumentos o no argumentos del texto, luego se clasificarán las
cláusulas según el rol que jueguen y por último se observará la relación existente 
entre dichas cláusulas.

\keywords{First keyword  \and Second keyword \and Another keyword.}
\end{abstract}
%
%
%

\section{Introducción}

\section{Problema a Resolver}


\section{Modelación del Problema}

\subsection{Secuencia por Secuencia}

\subsection{Convolución}

\subsection{Long Short-Term Memory (LSTM)}

\subsection{Conditional Random Field (CRF)}


\section{Implementación}
\subsection{Tensor Flow}


\section{Entrenamiento}
\subsection{Corpus}
\subsection{ Hyperparametros y Optimización }

\section{Evaluación}
\section{Resultados}

\section{Concluciones}

\section{Referencias}



\end{document}
